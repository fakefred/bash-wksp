% !TeX root = bash.tex
\section{V. Bash Scripting}
\begin{frame}[fragile]
\frametitle{Bash is a programming language}
Let's say you want to print many files at once.
\newline \newline
Try this simple example:
\begin{lstlisting}[language=bash]
$ for i in {01..05}; do \
    cat um-logo-$i.txt \
  done
\end{lstlisting}
\end{frame}

\begin{frame}[fragile]
\frametitle{Variables}
A variable in bash is defined this way:
\begin{lstlisting}[language=bash]
$ i=0
$ s='s'
\end{lstlisting}
Surprisingly these do \textbf{not} work:
\begin{lstlisting}[language=bash]
$ i = 0   # WRONG
$ s = 's' # WRONG
\end{lstlisting}
They can be accessed with a dollar sign:
\begin{lstlisting}[language=bash]
$ echo $i  # 0
$ echo $s  # s
\end{lstlisting}
\end{frame}

\begin{frame}[fragile]
\frametitle{Environment variables}
An \textbf{environment variable} can be set and accessed like this:
\begin{lstlisting}[language=bash]
$ export VAR=VALUE  # set
$ echo $VAR         # access
\end{lstlisting}
\tt{export} exposes the variable to programs that are not part of the shell
\footnote{aka not child processes}, such as Python.
\begin{block}{Note}
    If an environment variable works for utility A, it may or may not work
    for utility B. Refer to documentation when in doubt.
\end{block}
\end{frame}

\begin{frame}[fragile]
\frametitle{Environment variables}
Let's say you're downloading something from a completely legal website,
and you want the traffic to go through your completely legal local proxy
for completely legal reasons.
\begin{lstlisting}[language=bash]
# set environment variable for local proxy
$ export HTTPS_PROXY=http://localhost:8080/
# download the thing
$ curl -O https://legal.website/legal-thing
\end{lstlisting}
\end{frame}

\begin{frame}[fragile]
\frametitle{Shell scripts}
Open your favorite text editor and edit \tt{05-scripting/um.sh}:
\begin{lstlisting}[language=bash]
# 05-scripting/um.sh
for i in {01..05}; do
    cat um-logo-$i.txt
done
\end{lstlisting}
Save file, then come back to bash, \tt{cd} into \tt{05-scripting} and run:
\begin{lstlisting}[language=bash]
$ bash um.sh
\end{lstlisting}
\end{frame}

\begin{frame}
\frametitle{Exit status}
When a program exits, it emits an \textbf{exit status} (also called exit code).
By convention, an exit status of 0 implies success, and everything else means
something went wrong (consult respective man pages).
\newline \newline
For this very reason, in bash, \textbf{0 is boolean true} and
\textbf{everything else is false}.

\begin{block}{Note}
    When you \tt{return 0;} at the end of \tt{int main()}, you are
    emitting an exit status of zero.
\end{block}
\end{frame}

\begin{frame}[fragile]
\frametitle{\textbf{if} statements}
Usually, we use an \textbf{if} statement to:
\begin{itemize}
    \item Run a command and see if it succeeds
    \item Test the value of a variable
    \item Check if a file exists
\end{itemize}
It looks like this:
\begin{lstlisting}[language=bash]
if CONDITION; then
    BODY
elif CONDITION; then
    BODY
fi
\end{lstlisting}
\end{frame}

\begin{frame}[fragile]
\frametitle{\textbf{if} statements: exit code}
\begin{lstlisting}[language=bash]
# if-prog.sh
if mkdir temp; then
    # write to file only if mkdir emits exit code 0
    echo "temporary dir created" >> temp/log
fi
\end{lstlisting}
\end{frame}

\begin{frame}[fragile]
\frametitle{\textbf{if} statements: compare integers}
\begin{lstlisting}[language=bash]
# if-comp.sh
i=3
if [[ $i -eq 3 ]]; then
    echo "i is 3"
fi
\end{lstlisting}
Other operators:
\begin{table}
    \centering
    \begin{tabular}{ll}
        \tt{-ge} & $\geq$ \\
        \tt{-gt} & $>$    \\
        \tt{-le} & $\leq$ \\
        \tt{-lt} & $<$    \\
        \tt{-ne} & $\neq$
    \end{tabular}
\end{table}
\begin{block}{Note}
    Spaces after \verb|[[| and before \verb|]]| are \textbf{required}.
    \footnote{
        Trivia: In bash the \tt{[[ ]]} is built-in, but in some shells
        it's an executable called \tt{/usr/bin/[[}.
        See \url{https://serverfault.com/questions/138951/what-is-usr-bin}
    }
\end{block}
\end{frame}

\begin{frame}[fragile]
\frametitle{\textbf{if} statements: arithmetic}
\begin{lstlisting}[language=bash]
# if-arith.sh
i=3
if (( $i % 2 == 1 )); then
    echo "i is odd"
fi
\end{lstlisting}
\begin{block}{Note}
    We do not use \verb|-eq|, \verb|-gt| etc.~inside \verb|(( ))|.
\end{block}
\end{frame}

\begin{frame}[fragile]
\frametitle{\textbf{if} statements: test string variable}
\begin{lstlisting}[language=bash]
# if-str.sh
str="something"
if [[ -z $str ]]; then
    echo "str is empty"
elif [[ $str = 'something' ]]; then
    # = and == are both ok
    echo "str is 'something'"
fi
\end{lstlisting}
Complementary operators to \tt{-z} and \tt{=}:
\begin{table}
    \centering
    \begin{tabular}{ll}
        \tt{-n} & not empty \\
        \tt{!=} & not equal to
    \end{tabular}
\end{table}
\end{frame}

\begin{frame}[fragile]
\frametitle{\textbf{if} statements: file exists}
\begin{lstlisting}[language=bash]
# if-file.sh
file="something.txt"
if [[ -e $file ]]; then
    echo "$file exists"
fi
\end{lstlisting}
Common counterparts to \tt{-e}:
\begin{table}
    \centering
    \begin{tabular}{ll}
        \tt{-f} & exists and is regular file \\
        \tt{-d} & exists and is directory
    \end{tabular}
\end{table}
\end{frame}

\begin{frame}[fragile]
\frametitle{\textbf{for} statements}
The \tt{for} statement is usually used to:
\begin{itemize}
    \item Iterate over a range
    \item Iterate over a list of strings
\end{itemize}
It looks like:
\begin{lstlisting}[language=bash]
for VAR in LIST; do
    BODY
done
\end{lstlisting}
\end{frame}

\begin{frame}[fragile]
\frametitle{\textbf{for} statements: range}
\begin{lstlisting}[language=bash]
# for-timer.sh
for t in {10..1}; do
    echo "$t seconds left"
    sleep 1
done
\end{lstlisting}
\begin{block}{Note}
    In \textbf{double quotes}, variables will be expanded.
\end{block}
\end{frame}

\begin{frame}[fragile]
\frametitle{\textbf{for} statements: list of strings}
A string is split into a list with respect to whitespace.
\newline \newline
Let's say you want to create a backup of every file inside current directory:
\begin{lstlisting}[language=bash]
# for-backup.sh
for file in $(ls); do
    cp $file $file.backup
done
\end{lstlisting}
\begin{block}{Explanation}
    \verb|$()| executes a command and takes its output.
    It is called ``command substitution''.
\end{block}
\begin{block}{Observation}
    Why does bash think ``My Documents'' are two files?
\end{block}
\end{frame}

\begin{frame}[fragile]
\frametitle{\textbf{for} statements: IFS}
Bash splits strings on spaces, tabs, and newlines.
This is why \tt{My Documents} was split into \tt{My} and \tt{Documents}.
\newline \newline
Adjust the \textbf{IFS} environment variable to delimit on \verb|\n| only:
\begin{lstlisting}[language=bash]
# for-backup.sh
IFS=$'\n'
for file in $(ls); do
    cp $file $file.backup
done
\end{lstlisting}
\begin{block}{Explanation}
    \verb|$''| is called ``ANSI-C quoting''. It is not used very often.
    \footnote{
        More on this:
        \url{https://www.gnu.org/software/bash/manual/html_node/ANSI_002dC-Quoting.html}
    }
\end{block}
\end{frame}

\begin{frame}[fragile]
\frametitle{\textbf{for} statements: accumulator}
Let's try keeping count with an integer variable.
\begin{lstlisting}[language=bash]
# for-backup.sh
IFS=$'\n'
count=0
for file in $(ls); do
    cp $file $file.backup
    count=$(( count + 1 ))
done
echo "$count files backed up"
\end{lstlisting}
\begin{block}{Explanation}
    \verb|$(())| is called ``arithmetic expansion''.
\end{block}
\end{frame}

\begin{frame}[fragile]
\frametitle{Your turn}
Print the odd numbered lines of \tt{05-scripting/logos.txt}.
% \pause
% \begin{lstlisting}[language=bash]
% # logos.sh
% IFS=$'\n'
% count=0
% for line in $(cat logos.txt); do
    % count=$(( count + 1 ))
    % if (( count % 2 == 1 )); then
        % echo $line
    % fi
% done
% \end{lstlisting}
\end{frame}

\begin{frame}[fragile]
\frametitle{The Ultimate Challenge}
Each line in \tt{05-scripting/obf.txt} begins with an 8-digit number.
Print every line with its number greater than any of the lines above.
% \pause
% \begin{block}{Solution}
% \begin{lstlisting}[language=bash]
% IFS=$'\n'
% max=0
% for line in $(cat obf.txt); do
    % year=$(echo $line | grep -oE '^[0-9]{8}')
    % if [[ $year -gt $max ]]; then
        % echo $line
        % max=$year
    % fi
% done
% \end{lstlisting}
% \end{block}
\end{frame}

\begin{frame}
\frametitle{Conclusion}
\begin{itemize}
    \item Regex matches apparent patterns in strings
    \item Many tools support regex, but beware of standards
    \item Bash scripts are for basic automation
    \item When unwieldy, consider alternatives
\end{itemize}
\end{frame}
