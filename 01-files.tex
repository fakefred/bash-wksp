% !TeX root = bash.tex
\section{I. Files \& File Tree}
\begin{frame}[fragile]
\frametitle{Files}
Each of these is a different \textbf{file}:
\begin{itemize}
    \item \tt{a}
    \item \tt{.a} (Hidden)
    \item \tt{a.txt}
    \item \tt{A.txt}
    \item \tt{A.TXT}
\end{itemize}

\begin{block}{Note}
    The dot and suffix are part of the filename.
    \newline \newline
    \textbf{Avoid spaces and special characters} (except \verb|._-|).
    If you have to, surround filename in quotes:
    \tt{`Lab Report (3) final FINAL-1.docx'}
\end{block}
\end{frame}

\begin{frame}[fragile]
\frametitle{\tt{cat}: Printing a file}
Try this inside \tt{01-files/}:
\begin{lstlisting}[language=bash]
$ cat a
$ cat .a
$ cat a.txt
\end{lstlisting}
\begin{block}{Explanation}
    \tt{cat} is short for ``concatenate'' (to join together) but it's
    mostly used to print files.
\end{block}
\end{frame}

\begin{frame}[fragile]
\frametitle{\tt{cp, mv, rm}: Relocating a file}
Try this inside \tt{01-files/}:
\begin{lstlisting}[language=bash]
$ ls
$ cp a b
$ ls
$ mv a.txt b.txt
$ ls
$ rm b
$ ls
\end{lstlisting}
\begin{block}{Explanation}
    \begin{itemize}
        \item \tt{ls} lists files
        \item \textbf{copy} \tt{a} into a file called \tt{b}
        \item \textbf{move} \tt{a.txt} into a file called \tt{b.txt}
        \item \textbf{remove} \tt{b}
    \end{itemize}
\end{block}
\end{frame}

\begin{frame}[fragile]
\frametitle{\tt{cp, mv}: Overwriting and renaming}
When the destination does not exist, \tt{cp} and \tt{mv} simply create
that file. \textbf{Otherwise, it is destroyed and overwritten.}
\newline \newline
Try this inside \tt{01-files/}:
\begin{lstlisting}[language=bash]
$ cp a b      # creates b
$ cp a a.txt  # overwrites a.txt
\end{lstlisting}

Renaming a file in bash works like so:
\begin{lstlisting}[language=bash]
$ mv b newb
\end{lstlisting}
\end{frame}

\begin{frame}
\frametitle{Directories}
Each of these is a \textbf{directory} (``dir'' for short):
\begin{itemize}
    \item \tt{01-files/}
    \item \tt{01-files/c/}
    \item \tt{01-files/.c/} (Hidden dir)
\end{itemize}

\begin{block}{Convention}
    For clarity, we add a slash (\tt{/}) to the end of a directory in the
    slides. However, in reality it often makes no difference.
\end{block}
\end{frame}

\begin{frame}[fragile]
\frametitle{\tt{cd, pwd}: Changing directory}
Try this inside \tt{01-files/}:
\begin{lstlisting}[language=bash]
$ cd c/
$ pwd
$ cd ../
$ pwd
\end{lstlisting}
\begin{block}{Explanation}
    \begin{itemize}
        \item \tt{cd}: ``change directory''
        \item \tt{pwd}: ``print working directory''
        \item \tt{../} means ``parent directory''
    \end{itemize}
\end{block}
\end{frame}

\begin{frame}[fragile]
\frametitle{\tt{ls}: Listing directories}
Try this inside \tt{01-files/}:
\begin{lstlisting}[language=bash]
$ ls
$ ls -a
$ ls -l
$ ls -la
$ ls c/
\end{lstlisting}
\begin{block}{Explanation}
    \begin{itemize}
        \item \tt{ls}: ``list''
        \item \tt{-a} is short for \tt{--all}
        \item \tt{-l} enables long listing format
        \item \tt{-la} = \tt{-l} + \tt{-a}
    \end{itemize}
\end{block}
\end{frame}

\begin{frame}[fragile]
\frametitle{\tt{mkdir, rm}: Creating and deleting directories}
Try this inside \tt{01-files/}:
\begin{lstlisting}[language=bash]
$ mkdir dir/
$ rm -r dir/
$ mkdir -p dir/subdir/
$ ls dir/
\end{lstlisting}
\begin{block}{Explanation}
    \begin{itemize}
        \item \tt{mkdir}: ``make directory''
        \item \tt{-r} is short for \tt{--recursive}
        \item \tt{-p} is short for \tt{--parents}
    \end{itemize}
\end{block}
\end{frame}

\begin{frame}[fragile]
\frametitle{\tt{cp, mv}: Into and out of directories}
Try this inside \tt{01-files/}:
\begin{lstlisting}[language=bash]
$ cp a dir/
$ mv b.txt dir/
$ mv dir/a b.txt
\end{lstlisting}
\begin{block}{Explanation}
    \begin{itemize}
        \item Copy \tt{a} into \tt{dir/}
        \item Move \tt{b.txt} into \tt{dir/}
        \item Move \tt{dir/a} back into \tt{b.txt}
    \end{itemize}
\end{block}
\end{frame}

\begin{frame}
\frametitle{File tree}
Think of any directory as a tree.

\begin{figure}[h]
    \centering
    \begin{tikzpicture}
        \tikzstyle{every node}=[fill=blue!20,rounded corners]
        \tikzstyle{edge from parent}=[blue!50,thick,draw]
        \node {01-files/}[edge from parent fork down]
            child {node {a.txt}}
            child {
                node {dir/}
                    child {node {subdir/}}
                    child {node {b.txt}}
            };
    \end{tikzpicture}
\end{figure}
\end{frame}

\begin{frame}
\frametitle{Paths}
File $\cup$ directory = \textbf{path}.
\footnote{At least in the scope of this workshop.}
\newline \newline

No paths under the same directory can bear the same name.
These \textbf{cannot} coexist:
\begin{itemize}
    \item \tt{01-files/data/}, a directory
    \item \tt{01-files/data}, a regular file
\end{itemize}
\end{frame}

\begin{frame}
\frametitle{Absolute \& relative paths}
\begin{itemize}
    \item Paths beginning with \tt{/} are absolute: \tt{/usr/bin/cat}
    \item Otherwise it is relative: \tt{01-files/}
\end{itemize}

If you know where you are, you can convert a relative path to an absolute one.
\begin{example}
    Your location: \tt{/home/you/} \newline
    Relative path: \tt{bash-workshop/01-files/} \newline
    Absolute path: \tt{/home/you/bash-workshop/01-files/}
\end{example}
\end{frame}

\begin{frame}
\frametitle{Wildcard}
\tt{*} is a character to match any number of (including zero) characters.

\begin{block}{Exception}
    Hidden paths will remain hidden unless you explicitly specify the dot:
    \tt{.*}
\end{block}

\begin{example}
    \begin{table}
        \centering
        \begin{tabular}{lcccc}
                       & \tt{a/}    & \tt{b/}    & \tt{a-copy.txt} & \tt{b.txt} \\ \hline
            \tt{*}     & \checkmark & \checkmark & \checkmark      & \checkmark \\
            \tt{a*}    & \checkmark &            & \checkmark      &            \\
            \tt{*.txt} &            &            & \checkmark      & \checkmark \\
        \end{tabular}
    \end{table}
\end{example}

\scriptsize{Technically it's called a glob pattern but who cares. Also there are other
weird symbols like \tt{?} or \tt{[]} but I swear \tt{*} is most of
us will ever use.}
\end{frame}

\begin{frame}
\frametitle{\tt{.} and \tt{..}}
Inside every dir\footnote{Except \tt{/}} there are two special dirs:
\begin{itemize}
    \item \tt{./} — current dir
    \item \tt{../} — parent dir
\end{itemize}

You can use them in relative paths.
\begin{example}
    Your location: \tt{/home/you/} \newline
    Relative path: \tt{../friend/bash-workshop/01-files/} \newline
    (Note that \tt{/home/you/../friend/} is just \tt{/home/friend/}) \newline
    Absolute path: \tt{/home/friend/bash-workshop/01-files/}
\end{example}
\end{frame}

\begin{frame}
\frametitle{Challenge}
Inside \tt{01-files/}:
\begin{itemize}
    \item Enter \tt{challenge/}
    \item Create \tt{backup/}
    \item Copy \tt{a.txt} into \tt{dir/}
    \item Move \tt{dir/} into \tt{backup/}
    \item Verify using \tt{ls}
    \item Delete \tt{backup/}
\end{itemize}
\end{frame}

\begin{frame}<presentation:0>[fragile]
\frametitle{Solution}
\begin{lstlisting}[language=bash]
$ cd challenge/
$ mkdir backup/
$ cp a.txt dir/
$ mv dir/ backup/
$ ls
# Output: backup/  a.txt
$ ls backup/dir/
# Output: a.txt
$ rm -r backup/
\end{lstlisting}
\end{frame}
