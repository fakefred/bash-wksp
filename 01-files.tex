% !TeX root = bash.tex
\section{I. Files \& File Tree}
\begin{frame}[fragile]
\frametitle{Files}
Each of these is a different \textbf{file}:
\begin{itemize}
    \item \tt{a}
    \item \tt{.a} (Hidden)
    \item \tt{a.txt}
    \item \tt{A.txt}
    \item \tt{A.TXT}
\end{itemize}

\begin{block}{Note}
    The dot and suffix are part of the filename.
\end{block}

\begin{block}{Tip}
    For convenience, \textbf{avoid spaces} and most of the special characters
    (except \verb|._-|) when working in bash. If you have to, surround filename
    in quotes: \tt{`Lab Report (3) final FINAL-1.docx'}
\end{block}
\end{frame}

\begin{frame}[fragile]
    \frametitle{\tt{cat}: Printing a file}
Try this inside \tt{01-files/}:
\begin{lstlisting}[language=bash]
$ cat a
$ cat .a
$ cat a.txt
\end{lstlisting}
\begin{block}{Explanation}
    \tt{cat} is short for ``concatenate'' (to join together) but it's
    mostly used to print files.
\end{block}
\end{frame}

\begin{frame}[fragile]
\frametitle{\tt{cp, mv, rm}: Relocating a file}
Try this inside \tt{01-files/}:
\begin{lstlisting}[language=bash]
$ cp a b
$ mv a.txt b.txt
$ rm b
\end{lstlisting}
\begin{block}{Explanation}
    \begin{itemize}
        \item ``copy'' \tt{a} into a file called \tt{b}
        \item ``move'' \tt{a.txt} into a file called \tt{b.txt}
        \item ``remove'' \tt{b}
    \end{itemize}
\end{block}
\end{frame}

\begin{frame}[fragile]
\frametitle{\tt{cp, mv}: Overwriting and renaming}
When the destination does not exist, \tt{cp} and \tt{mv} simply create
that file. \textbf{Otherwise, it is destroyed and overwritten.}
\newline \newline
Try this inside \tt{01-files/}:
\begin{lstlisting}[language=bash]
$ cp a b      # b is created
$ cp a a.txt  # a.txt is overwritten
\end{lstlisting}

When you \tt{mv} a file into a file that doesn't exist, you are essentially
\textbf{renaming} it. Recall this command from last slide:
\begin{lstlisting}[language=bash]
$ mv a.txt b.txt
\end{lstlisting}
\end{frame}

\begin{frame}
\frametitle{Directories}
Each of these is a \textbf{directory} (``dir'' for short):
\begin{itemize}
    \item \tt{dir/}
    \item \tt{01-files/}
    \item \tt{01-files/dir/}
    \item \tt{01-files/.dir/} (Hidden dir)
\end{itemize}

\begin{block}{Convention}
    For clarity, we add a slash (\tt{/}) to the end of a directory in the
    slides.    However, in reality it often makes no difference.
\end{block}
\end{frame}

\begin{frame}[fragile]
\frametitle{\tt{cd, pwd}: Changing directory}
Try this inside \tt{01-files/}:
\begin{lstlisting}[language=bash]
$ cd dir/
$ pwd
$ cd ../
$ pwd
\end{lstlisting}
\begin{block}{Explanation}
    \begin{itemize}
        \item \tt{cd} is short for ``change directory''
        \item \tt{pwd} is short for ``print working directory''
        \item \tt{..} is a special directory. It is the ``parent''
    \end{itemize}
\end{block}
\end{frame}

\begin{frame}[fragile]
\frametitle{\tt{ls}: Listing directories}
Try this inside \tt{01-files/}:
\begin{lstlisting}[language=bash]
$ ls
$ ls -a
$ ls -l
$ ls -a -l
$ ls dir/
\end{lstlisting}
\begin{block}{Explanation}
    \begin{itemize}
        \item \tt{ls} is short for ``list''
        \item \tt{-a} is short for \tt{--all}
        \item \tt{-l} enables long listing format
    \end{itemize}
\end{block}
\end{frame}

\begin{frame}[fragile]
\frametitle{\tt{mkdir, rm}: Creating and deleting directories}
Try this inside \tt{01-files/}:
\begin{lstlisting}[language=bash]
$ mkdir dir/
$ rm -r dir/
$ mkdir -p dir/subdir/
$ ls dir/
\end{lstlisting}
\begin{block}{Explanation}
    \begin{itemize}
        \item \tt{mkdir} stands for ``make directory''
        \item \tt{-r} is short for \tt{--recursive}
        \item \tt{-p} is short for \tt{--parents}
    \end{itemize}
\end{block}
\end{frame}

\begin{frame}[fragile]
\frametitle{\tt{cp, mv}: Into and out of directories}
Try this inside \tt{01-files/}:
\begin{lstlisting}[language=bash]
$ cp a dir/
$ mv b.txt dir/
$ mv b.txt dir/a
$ mv dir/a b.txt
\end{lstlisting}
\begin{block}{Explanation}
    \begin{itemize}
        \item Line 1 copies \tt{a} into \tt{dir/}
        \item Line 2 moves \tt{b.txt} into \tt{dir/}
        \item Line 3 moves \tt{b.txt} into \tt{dir/} and overwrite
            \tt{dir/a}
        \item Line 4 moves \tt{dir/a} back into \tt{b.txt}
    \end{itemize}
\end{block}
\end{frame}

\begin{frame}
\frametitle{File tree}
Think of any directory as a tree.

\begin{figure}[h]
    \centering
    \begin{tikzpicture}
        \tikzstyle{every node}=[fill=blue!20,rounded corners]
        \tikzstyle{edge from parent}=[blue!50,thick,draw]
        \node {01-files/}[edge from parent fork down]
            child {node {a.txt}}
            child {
                node {dir/}
                    child {node {subdir/}}
                    child {node {b.txt}}
            };
    \end{tikzpicture}
\end{figure}
\end{frame}

\begin{frame}
\frametitle{Paths}
File $\cup$ directory = \textbf{path}.
\footnote{At least in the scope of this workshop.}
\newline \newline

No paths under the same directory can bear the same name.
These \textbf{cannot} coexist:
\begin{itemize}
    \item \tt{01-files/data/}, a directory
    \item \tt{01-files/data}, a regular file
\end{itemize}
\end{frame}

\begin{frame}
\frametitle{Absolute \& relative paths}
\begin{itemize}
    \item Paths beginning with \tt{/} are absolute: \tt{/usr/bin/cat}
    \item Otherwise it is relative: \tt{01-files/}
\end{itemize}

If you know where you are, you can convert a relative path to an absolute one.
\begin{example}
    Your location: \tt{/home/you/} \newline
    Relative path: \tt{bash-workshop/01-files/} \newline
    Absolute path: \tt{/home/you/bash-workshop/01-files/}
\end{example}
\end{frame}

\begin{frame}
\frametitle{Wildcard}
\tt{*} is a character to match any number of (including zero) characters.

\begin{block}{Exception}
    Hidden paths will remain hidden unless you explicitly specify the dot:
    \tt{.*}
\end{block}

\begin{example}
    Consider a dir containing \tt{a/}, \tt{b/}, \tt{a-copy.txt} and
    \tt{b.txt}.
    \begin{itemize}
        \item \tt{*} matches \tt{a/ b/ a-copy.txt b.txt}
        \item \tt{a*} matches \tt{a/ a-copy.txt}
        \item \tt{*.txt} matches \tt{a-copy.txt b.txt}
    \end{itemize}
\end{example}

\scriptsize{Technically it's called a glob pattern but who cares. Also there are other
weird symbols like \tt{?} or \tt{[]} but I swear \tt{*} is most of
us will ever use.}
\end{frame}

\begin{frame}
\frametitle{\tt{.} and \tt{..}}
Inside every dir\footnote{Except \tt{/}} there are two special dirs:
\begin{itemize}
    \item \tt{.} — current dir
    \item \tt{..} — parent dir
\end{itemize}

You can use them in relative paths.
\begin{example}
    Your location: \tt{/home/you/} \newline
    Relative path: \tt{../friend/bash-workshop/01-files/} \newline
    (Note that \tt{/home/you/../friend/} is just \tt{/home/friend/}) \newline
    Absolute path: \tt{/home/friend/bash-workshop/01-files/}
\end{example}
\end{frame}

\begin{frame}<presentation:0>[fragile]
\frametitle{Quiz: Expand relative paths}
Suppose your home directory is \tt{/home/sjtu/} and you are currently in
\tt{/home/sjtu/lbl/}. Expand these relative paths to absolute ones:
\begin{itemize}
    \item \tt{lobby}
    \item \tt{3f/ylm}
    \item \tt{./elevator}
    \item \tt{../library/}
    \item \tt{../../fdu/}
    \item \verb|~/lawson| % \tt doesn't format ~ well
\end{itemize}
\end{frame}

\begin{frame}<presentation:0> % hide frame
\frametitle{Answer}
\begin{itemize}
    \item \tt{/home/sjtu/lbl/lobby}
    \item \tt{/home/sjtu/lbl/3f/ylm}
    \item \tt{/home/sjtu/lbl/elevator}
    \item \tt{/home/sjtu/library/}
    \item \tt{/home/fdu/}
    \item \tt{/home/sjtu/lawson}
\end{itemize}
\end{frame}

\begin{frame}<presentation:0>
\frametitle{File commands cheatsheet (part 1)}
% TODO pwd
\begin{itemize}
    \item \tt{cat FILE} — print out FILE
    \item \tt{cd DIR/} — change current directory to DIR
    \begin{itemize}
        \item \tt{cd} — go to home dir
    \end{itemize}
    \item \tt{ls DIR/} — list dirs and files in DIR
    \begin{itemize}
        \item \tt{ls} — list current dir
    \end{itemize}
    \item \tt{pwd} — print where you are
    \item \tt{mkdir DIR/} — create dir named DIR
    \item \tt{rm FILE} — remove FILE
    \begin{itemize}
        \item \tt{rm -r DIR/} — remove everything in DIR and itself
        \item \tt{rm -r DIR/*} — remove everything in DIR but not DIR
    \end{itemize}
\end{itemize}
\end{frame}

\begin{frame}<presentation:0>
\frametitle{File commands cheatsheet (part 2)}
\begin{itemize}
    \item \tt{cp SRC DEST} — copy file named SRC to DEST
    \begin{itemize}
        \item \tt{cp SRC DEST\_DIR/} — copy SRC into DEST\_DIR/
        \item \tt{cp -r SRC\_DIR/ DEST\_DIR/}
            — copy SRC\_DIR/ and everything inside into DEST\_DIR
    \end{itemize}
    \item \tt{mv SRC DEST} — move (rename) SRC to DEST
    \begin{itemize}
        \item \tt{mv SRC DEST\_DIR/} — move SRC into DEST\_DIR
        \item \tt{mv SRC\_DIR/ DEST\_DIR/}
            — move SRC\_DIR into DEST\_DIR (no \tt{-r} needed)
    \end{itemize}
\end{itemize}
\begin{alertblock}{Warning}
    \tt{cp} and \tt{mv} will overwrite your files by default.
    How to avoid that? Check your cheatsheet.
\end{alertblock}
\end{frame}

\begin{frame}
\frametitle{Task}
Inside \tt{01-files/}:
\begin{itemize}
    \item Enter \tt{task/}
    \item Create \tt{backup/}
    \item Copy \tt{a.txt} into \tt{dir/}
    \item Move \tt{dir/} into \tt{backup/}
    \item Verify if you succeeded using \tt{ls}
    \item Delete \tt{backup/}
\end{itemize}
\end{frame}

\begin{frame}<presentation:0>[fragile]
\frametitle{Solution}
\begin{lstlisting}[language=bash]
$ mkdir backup/
$ cp a.txt dir/
$ mv dir/ backup/
$ ls
# Output: backup/  a.txt
$ ls backup/dir/
# Output: a.txt
$ rm -r backup/
\end{lstlisting}
\end{frame}
