% !TeX root = bash.tex
\section{IV. Regex}
\begin{frame}
\frametitle{What's a regex}
\textbf{Regex} is a portmanteau of ``regular expressions''.
It is a string to search for patterns in text. \newline

It is so powerful that text cannot define it. \newline

However, it is not fit for every purpose, the same way you wouldn't
drive a car to visit your downstair neighbor, a Hawaiian volcano, or
the edge of the universe.\newline

Use it wisely.
\end{frame}

\begin{frame}[fragile]
\frametitle{What can (and can't) a regex do}
A regex can (or at least attempt to):
\begin{itemize}
    \item Match a cell phone number: \verb|[0-9]{11}|
    \item Match a .com domain: \verb|([A-Za-z0-9_\-]+\.)+com|
    \item Match \textit{Star Wars} subtitles but not \textit{Star Trek}:
        \verb!m | [tn]|b!\footnote{Credit: \url{https://xkcd.com/1313/}}
\end{itemize}
It cannnot:
\begin{itemize}
    \item Understand emotion (you want NLP)
    \item Generate AI art (you want stable diffusion)
    \item Moderate a Minecraft server (you want a real human)
\end{itemize}
\end{frame}

\begin{frame}
\frametitle{When to use regex?}
Where a simple substring match won't do, and a program is an overkill, regex is
your friend. \newline

Imagine you want to find an email by subject.
\begin{itemize}
    \item Case 1: you want everything that contains ``VV256''. Solution: just
        search for ``VV256''.
    \item Case 2: you want emails written in Spanish. You might want to use
        some online translation APIs and definitely need to write code.
    \item Case 3: you want emails that contain a JI course code in the ECE
        category. This calls for regex.
\end{itemize}
\end{frame}

\begin{frame}
\frametitle{Regex is a mess}
\begin{block}{Quote}
I define UNIX as 30 definitions of regular expressions living under one roof.
\begin{flushright}
    — Donald Knuth\footnote{Digital Typography, ch. 33, p. 649 (1999)}
\end{flushright}
\end{block}

Of all the standards, two stand dominant:
\textbf{ERE} (Extended RegEx, POSIX compliant), and
\textbf{PCRE} (Perl Compatible RegEx, used in Perl and Python).
Our focus today will be ERE.
\end{frame}

\begin{frame}[fragile]
\frametitle{Common patterns (part 1)}
\begin{table}
    \centering
    \begin{tabular}{ll}
        \verb|.|            & any character \\
        \verb|\.|           & a literal dot \\
        \verb|[aeiou]|      & any vowel \\
        \verb|[^aeiou]|     & anything but a vowel \\
        \verb|[0-9]|        & any digit \\
        \verb|[A-Za-z]|     & any letter \\
        \verb|^|            & beginning of line \\
        \verb|$|            & end of line \\
        \verb|A?|           & zero or one A \\
        \verb|A+|           & one or more A's (\emph{Positive closure}) \\
        \verb|A*|           & zero or more A's (\emph{Kleene closure}) \\
        \verb|A{,6}|        & up to six A's \\
        \verb|A{4,6}|       & four to six A's \\
        \verb!(ls|cd|rm)!   & one of ls, cd, and rm
    \end{tabular}
\end{table}
\end{frame}

\begin{frame}[fragile]
\frametitle{Common patterns (part 2)}
\begin{table}
    \centering
    \begin{tabular}{ll}
        \verb|\w|           & \verb|[A-Za-z0-9_]| \\
        \verb|\W|           & anything that does not match \verb|\w| \\
        \verb|\s|           & whitespace (space, tab, linebreak, etc) \\
        \verb|\S|           & anything but whitespace \\
        \verb|\b|           & boundary of a word \\
        \verb|\B|           & you guessed it
    \end{tabular}
\end{table}
\end{frame}

\begin{frame}[fragile]
\frametitle{Quiz: Does it match?}
What strings does this regex match? \newline

\Large \verb!(^cat|cat$)! \normalsize

\begin{itemize}
    \item \verb|cat|                      % yes
    \item \verb|^cat$|                    % no
    \item \verb|cats|                     % yes
    \item \verb|cat /etc/fstab|           % yes
    \item \verb|I have a cat.|            % no
    \item \verb|Cats are the best.|       % no
    \item \verb|Concatenate these files|  % no
\end{itemize}
\end{frame}

\begin{frame}[fragile]
\frametitle{Quiz: Does it match?}
What matches \tt{+86 021-54749110} but not \tt{+86021-54749110}?
\begin{itemize}
    \item \verb|\+86\s+[0-9]{3}-[0-9]{8}|
    \item \verb|\+86\s*[0-9]{3}-[0-9]{8}| % this one
\end{itemize}

What does \Large \verb|[um]jicanvas.com| \normalsize match?
\begin{itemize}
    \item \tt{umjicanvas.com}
    \item \tt{jicanvas.com}
    \item \tt{ujicanvasecom} % only this one
\end{itemize}

\begin{block}{Challenge\footnote{Want more challenges? Try \url{https://alf.nu/RegexGolf}!}}
    How to fix this regex? \pause \verb|(um)?jicanvas\.com|
\end{block}
\end{frame}

\begin{frame}[fragile]
\frametitle{But how to use a regex, anyway?}
Try this in \tt{04-regex/}:
\begin{lstlisting}[language=bash]
$ grep -E '.+\..+@sjtu.edu.cn' faculty
\end{lstlisting}
\pause
\begin{block}{Observation}
    The regex matches all email addresses in the file \tt{faculty}
    that look like ``firstname.lastname@sjtu.edu.cn''.
\end{block}
\begin{block}{Challenge}
    Can you come up with something shorter?
\end{block}
\end{frame}

\begin{frame}[fragile]
\frametitle{One more example}
\begin{lstlisting}[language=bash]
$ grep -o -E '^[^@]{,8}' faculty
\end{lstlisting}
\pause
\begin{block}{Observation}
    This regex truncates each line after the 8th character or before @,
    whichever comes first.
\end{block}
\begin{block}{Explanation}
    \begin{tabular}{ll}
        \verb|^|    & From beginning of each line \\
        \verb|[^@]| & Match all characters except @ \\
        \verb|{,8}| & Until we reach max length 8
    \end{tabular}
\end{block}
\end{frame}

\begin{frame}[fragile]
\frametitle{Your turn}
Extract all course codes from \tt{04-regex/courses}. \newline

\begin{tabular}{lcl}
    VG100 Introduction to Engineering & & VG100 \\
    VM020 Machineshop Training & $\Longrightarrow$ & VM020 \\
    VP140 Physics I & & VP140 \\
    VP141 Physics Lab I & & VP141
\end{tabular}
\pause
\begin{block}{Solution (one version)}
    \verb|grep -oE 'V[A-Z][0-9]+' courses|
\end{block}
\end{frame}

\begin{frame}[fragile]
\frametitle{Find and replace with \tt{sed}}
\tt{sed} is a powerful tool for transforming text.
We will be using one very specific syntax:
\begin{lstlisting}[language=bash]
$ COMMAND | sed 's/FIND/REPLACE/'
\end{lstlisting}
Use \tt{-E} for regex. This command redacts all the IPv4 addresses in the
file \tt{ipv4}:
\begin{lstlisting}[language=bash]
$ sed -E 's/([0-9]{1,3}\.){3}[0-9]{1,3}/redacted/g' \
    ipv4
\end{lstlisting}
\begin{block}{Note}
    Without the \tt{g} at the end, \tt{sed} will skip to the
    next line after only one replacement.
\end{block}
\end{frame}

\begin{frame}[fragile]
\frametitle{Groups in regex}
What if you only want to redact the subnet (i.e. last part) of the IP addresses?
\textbf{Capturing groups} will be useful.
\begin{lstlisting}[language=bash]
$ sed -E 's/(([0-9]{1,3}\.){3})[0-9]{1,3}/\1xxx/g' \
    ipv4
\end{lstlisting}
\end{frame}
